\documentclass[a4paper]{article}

\usepackage{a4wide}

\usepackage[colorlinks=true,linkcolor=black,urlcolor=blue,bookmarksopen=true]{hyperref}
\usepackage{bookmark}
\usepackage{fancyhdr}
\usepackage[spanish]{babel}
\usepackage[utf8]{inputenc}
\usepackage[T1]{fontenc}
\usepackage{graphicx}
\usepackage{float}
\usepackage{listings}


\pagestyle{fancy} % Encabezado y pie de página

\fancyhf{}
\fancyhead[C]{
	Universidad Nacional de Salta - Facultad de Cs Exactas \\
	Programación para LAS \& TUP\\
	\textit{Trabajo Práctico Nivel 8 Recursión con Vectores} \\
}
\fancyhead[R]{	Año: 2023}
\renewcommand{\headrulewidth}{0.4pt}
\fancyfoot[C]{\thepage}
\renewcommand{\footrulewidth}{0.4pt}





\begin{document}

\hfill

\section*{Clase Práctica
}
\subsection*{Ejercicio 1: Para las siguientes situaciones, realice un módulo recursivo y la traza con los 	datos indicados para cada una: }
\renewcommand{\theenumi}{\alph{enumi}}
\begin{enumerate}{}
	\item 	 Desarrollar un módulo recursivo que permita sumar los dígitos de un número natural X.	Realizar la traza para X = 12345.
	\item 	 Desarrollar un módulo recursivo que permita determinar si un dígito D pertenece a un número entero positivo N.Realizar la traza para D = 1 y X = 45673.
	\item  Desarrollar un modulo recursivorecursiva en C que calcula el término de la serie de Fibonacci para un índice dado.Realiza la traza para N=8 
	
	
\end{enumerate}
\subsection*{Ejercicio 2:  Desarrolla la traza del siguiente codigo en C que imprime los números primos menores que un número natural N.	Realizar la traza para N = 100, luego con N = 27.
}
\lstinputlisting{codigopractico2.c}
\subsection*{Ejercicio 3:  Se pide crear un programa recursivo que permita generar aleatoriamente una lista de N números enteros en el intervalo [A,B], posteriormente, ingresar un número determinar si es divisor de algunos de los numeros recientemente generados. Mostrar los numeros en los cuales es divisor.}

\subsection*{Ejercicio 4:  Definir función recursiva llamada numerologia, que devolverá un numero a partir de una fecha de nacimiento. El número se calcula sumando cada uno de los dígitos de la fecha completa: día, mes y año. Por ejemplo, si alguien nació el 26 de marzo de 1966, sumará: 2+6+0+3+1+9+6+6=33, se continúa reduciendo 3+3=6, el programa deberá devolver el numero 6 en este caso.	Pautas:	La fecha se ingresara  con el siguiente formato : 25/05/1979.	Deberá haber una funcion recursiva que sume.}

\newpage
\section*{Ejercicios Resueltos}
\subsection*{Ejercicio 1: Para las siguientes situaciones, realice un módulo recursivo y la traza con los 	datos indicados para cada una: }
\subsubsection*{a.- Desarrollar un módulo recursivo que permita sumar los dígitos de un número natural X.	Realizar la traza para X = 12345.}
\lstinputlisting{codigopractico1a.c}
\subsubsection*{b.- Desarrollar un módulo recursivo que permita determinar si un dígito D pertenece a un número entero positivo N.Realizar la traza para D = 1 y X = 45673.}
\lstinputlisting{buscarDigito.c}
\subsubsection*{c.-Desarrollar un modulo recursivorecursiva en C que calcula el término de la serie de Fibonacci para un índice dado. Realiza la traza para N=8.}
\lstinputlisting{codigopractico1c.c}
\subsection*{Ejercicio 3: Para las siguientes situaciones, realice un módulo recursivo y la traza con los 	datos indicados para cada una:}
\lstinputlisting{codigopractico2.c}
\subsection*{Ejercicio 4:  Definir función recursiva llamada numerologia, que devolverá un numero a partir de una fecha de nacimiento. El número se calcula sumando cada uno de los dígitos de la fecha completa: día, mes y año. Por ejemplo, si alguien nació el 26 de marzo de 1966, sumará: 2+6+0+3+1+9+6+6=33, se continúa reduciendo 3+3=6, el programa deberá devolver el numero 6 en este caso.	Pautas:	La fecha se ingresara  con el siguiente formato : 25/05/1979.	Deberá haber una funcion recursiva que sume.}
\lstinputlisting{numerologia.c}
\end{document}